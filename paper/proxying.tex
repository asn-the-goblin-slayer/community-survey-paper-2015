%%%%%%%%%%%%%%%%%%%%%%%%%%%%%%%%%%%%%%%%%%%%%%%%%%%%%%%%%%%%%%%%%%%%%%%%%%%%%%%%
% Community-driven Obfuscation Paper
%
% This is proxying.tex. Some background on proxying systems and their use in
% censorship-resistance systems (CRSs). 
%%%%%%%%%%%%%%%%%%%%%%%%%%%%%%%%%%%%%%%%%%%%%%%%%%%%%%%%%%%%%%%%%%%%%%%%%%%%%%%%
\section{Background on Proxying Systems}
\label{section:proxying}


\subsection{Overview of typical circumvention system architectures}

\subsection{Challenges faced by circumvention systems}



\paragraph{Client integrity.} 

\tcrnote{Ensuring secure distribution of client software}

\paragraph{Proxy integrity.}

\tcrnote{Ensuring enough honest proxies}

\paragraph{Secure rendezvous.}

\tcrnote{Finding proxy IPs without making IP blacklisting easy + key negotiation}

The next challenge that circumvention systems face is how to
distribute their proxies to users. This is problematic because it's
hard to distinguish between legitimate clients who are looking for
proxies and the adversary who is trying to enumerate and block these
proxies.

Some systems (e.g. meek/uproxy) avoid this problem by incorporating
the rendezvous inside their protocol, whereas other systems (obfs3,
fteproxy, etc.) ask users to acquire proxies using various
distribution systems that rely on rate limiting features like CAPTCHA
or the web of trust.

We consider this topic orthogonal to traffic obfuscation and will not
examine it further on this paper.

\paragraph{Network obfuscation.} 

\tcrnote{Ensuring tool can’t be recognized on wire
w/o IP blacklist) and clarification that investigating this will be the primary
focus of the paper.}



